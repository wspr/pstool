\RequirePackage{filecontents}
\begin{filecontents*}{README.txt}
__________________
The pstool package
v1.0

A package like "pst-pdf" for processing PostScript graphics 
with psfrag labels within pdfLaTeX documets. The difference 
with this package is that every graphic is compiled individually, 
drastically speeding up compilation time when only a single
figure needs re-processing.

Running `latex` on pstool.tex will produce the files
  pstool.ins, pstool.sty, and README.txt,
and compile the PDF documentation. 

Executing `tex pstool.ins` produces the files above 
except pstool.ins (and the PDF file, obviously).

Will Robertson & Zebb Prime
Copyright 2008
\end{filecontents*}

\begin{filecontents}{pstool.sty}
\ProvidesPackage{pstool}[2008/08/31 v1.0
  Wrapper for processing PostScript/psfrag figures]

% External packages
\RequirePackage{%
  catchfile,color,ifpdf,ifplatform,graphicx,psfrag,suffix,xkeyval}
\RequirePackage{inversepath}[2008/07/31 v0.2]

% Allocations
\newif\if@pstool@always@
\newif\if@pstool@never@
\newif\if@pstool@pdfcrop@
\newif\if@pstool@verbose@
\newwrite\pstool@out

% These are cute
\providecommand\OnlyIfFileExists[2]{\IfFileExists{#1}{#2}{}}
\providecommand\NotIfFileExists[2]{\IfFileExists{#1}{}{#2}}

% \subsection{Package options}

\define@choicekey*{pstool.sty}{crop}[\@tempa\@tempb]{preview,pdfcrop}{%
  \ifcase\@tempb\relax
    \@pstool@pdfcrop@false
  \or
    \@pstool@pdfcrop@true
  \or
  \fi
}

\define@choicekey*{pstool.sty}{process}[\@tempa\@tempb]{all,none,auto}{%
  \ifcase\@tempb\relax
    \@pstool@always@true
  \or
    \@pstool@never@true
  \or
  \fi
}

\define@choicekey*{pstool.sty}{mode}
  [\@tempa\@tempb]{errorstop,nonstop,batch}{%
    \ifnum\@tempb=2\relax
      \@pstool@verbose@false
    \else
      \@pstool@verbose@true
    \fi
    \edef\pstool@mode{\@tempa mode}%
}
\ExecuteOptionsX{mode=batch}

\DeclareOptionX{cleanup}{\def\pstool@rm@files{#1}}
\ExecuteOptionsX{cleanup={.tex,.dvi,.ps,.pdf,.log,.aux}}

\DeclareOptionX{suffix}{\def\pstool@suffix{#1}}
\ExecuteOptionsX{suffix={-pstool}}

\DeclareOptionX{latex-options}{\def\pstool@latex@opts{#1}}
\DeclareOptionX{dvips-options}{\def\pstool@dvips@opts{#1}}
\DeclareOptionX{ps2pdf-options}{\def\pstool@pspdf@opts{#1}}
\DeclareOptionX{pdfcrop-options}{\def\pstool@pdfcrop@opts{#1}}
\ExecuteOptionsX{%
  latex-options={},
  dvips-options={},
  ps2pdf-options={},
  pdfcrop-options={}}

\ifshellescape\else
  \ExecuteOptionsX{process=none}
  \PackageWarning{pstool}{^^J\space\space%
    Package option [process=none] activated^^J\space\space
    because -shell-escape is not enabled.^^J%
    This warning occurred}
\fi

\ProcessOptionsX

% \section{Macros}

% Used to echo information to the console output.
% Can't use \typeout because it's asynchronous with
% any |\immediate\write18| processes (for some reason).
\def\pstool@echo#1{%
  \if@pstool@verbose@
    \pstool@echo@verbose{#1}%
  \fi}

\def\pstool@echo@verbose#1{%
  \immediate\write18{echo "#1"}%
}

\let\pstool@includegraphics\includegraphics

% Command line abstractions between platforms:
\edef\pstool@cmdsep{\ifwindows\string&\else\string;\fi\space}
\edef\pstool@rm@cmd{\ifwindows del \else rm -- \fi}

% Delete a file if it exists (only the filename is supplied and the path is inferred):
\newcommand\pstool@rm[2]{%
  \OnlyIfFileExists{#1#2}{%
    \immediate\write18{%
      cd "#1"\pstool@cmdsep\pstool@rm@cmd "#2"}}%
}

% Generic function to execute a command on the shell and pass its exit status back into \LaTeX. Any number of \cmd\pstool@exe\ statements can be made consecutively followed by \cmd\pstool@endprocess, which also takes an argument. If \emph{any} of the shell calls failed, then the execution immediately skips to the end and expands \cmd\pstool@error\ instead of the argument to \cmd\pstool@endprocess.
%
% |#1|: `name' of process\\
% |#2|: relative path where to execute the command\\
% |#3|: the command itself
\newcommand\pstool@exe[3]{%
  \pstool@echo{^^J=== pstool: #1 ===}%
  \pstool@shellexecute{#2}{#3}%
  \pstool@retrievestatus{#2}%
  \ifnum\pstool@status > \z@
    \PackageWarning{pstool}{Execution failed during process:^^J\space\space#3^^JThis warning occurred}%
    \expandafter\pstool@abort
  \fi}

% Edit this definition to print something else when graphic processing fails.
\def\pstool@error{%
  \fbox{%
    \parbox{0.8\linewidth}{%
      \color{red}\raggedright\ttfamily\scshape\small
      An error occured processing graphic \upshape`\ip@directpath\ip@lastelement'}}}

\def\pstool@abort#1\pstool@endprocess{\pstool@error\@gobble}
\let\pstool@endprocess\@firstofone

% It is necessary while executing commands on the shell to write the exit status to a temporary file to test for failures in processing. (If all versions of |pdflatex| supported input pipes, things might be different.)
\def\pstool@shellexecute#1#2{%
  \immediate\write18{%
    cd "#1" \pstool@cmdsep
    #2 \pstool@cmdsep
    \ifwindows
       call echo
         \string^\@percentchar ERRORLEVEL\string^\@percentchar
    \else
       echo \detokenize{$?}
    \fi
    > \pstool@statusfile}%
% That's the execution; now we need to flush the write buffer to the status file. This ensures the file is written to disk properly (allowing it to be read by \cmd\CatchFileEdef). Not necessary on Windows, whose file writing is evidently more crude/immediate.
  \ifwindows\else
    \immediate\write18{%
      touch #1\pstool@statusfile}%
  \fi}
\def\pstool@statusfile{pstool-statusfile.txt}

% Read the exit status from the temporary file and delete it.\\
% |#1| is the path\\
% Status is recorded in \cmd\pstool@status.
\def\pstool@retrievestatus#1{%
  \CatchFileEdef{\pstool@status}{#1\pstool@statusfile}{}%
  \pstool@rm{#1}{\pstool@statusfile}%
  \ifx\pstool@status\pstool@statusfail
    \PackageWarning{pstool}{%
      Status of process unable to be determined:^^J  #1^^J%
      Trying to proceed... }%
    \def\pstool@status{0}%
  \fi}
\def\pstool@statusfail{\par }% what results when \TeX\ reads an empty file

% \subsection{File age detection}
% Use |ls| (or |dir|) to detect if the EPS is newer than the PDF.
\def\pstool@IfnewerEPS{%
  \edef\pstool@filenames{\ip@lastelement.eps\space \ip@lastelement.pdf\space}%
  \immediate\write18{%
    cd "\ip@directpath"\pstool@cmdsep
    \ifwindows
      dir /T:W /B /O-D "\ip@lastelement.eps" "\ip@lastelement.pdf" > \pstool@statusfile
    \else
      ls -t "\ip@lastelement.eps" "\ip@lastelement.pdf" > \pstool@statusfile
    \fi
  }%
  \pstool@retrievestatus{\ip@directpath}%
  \ifx\pstool@status\pstool@filenames
    \expandafter\@firstoftwo
  \else
    \expandafter\@secondoftwo
  \fi
}

% A wrapper for \cs{inversepath*}. Long story short, always need a relative path to a filename even if it's in the same directory.
\def\pstool@getpaths#1{%
  \edef\@tempa{\unexpanded{\inversepath*}{#1}}%
  \@tempa% calculate filename, path \& inverse path
  \ifx\ip@directpath\@empty
    \def\ip@directpath{./}%
  \fi
% Strip off a possible wayward |.eps| suffix.
  \edef\ip@lastelement{%
    \expandafter\pstool@stripEPS\ip@lastelement.eps\@nil
  }%
}

\def\pstool@stripEPS#1.eps#2\@nil{#1}
% |test.eps\@nil|->|test|\\
% |test.eps.eps\@nil|->|test|

% \section{Command parsing}
% User input is \cmd\pstool\ (with optional |*| or |!| suffix) which turns into one of the following three macros depending on the mode.

\newcommand\pstool@alwaysprocess[3][]{%
  \pstool@getpaths{#2}%
  \pstool@process{#1}{#3}}

\newcommand\pstool@neverprocess[3][]{%
  \pstool@includegraphics[#1]{#2}}

% For regular operation, which processes the figure only if
% the command is starred, or the PDF doesn't exist.
\newcommand\pstool@maybeprocess[3][]{%
  \pstool@getpaths{#2}%
  \IfFileExists{#2.pdf}{%
    \pstool@IfnewerEPS{% needs info from |\pstool@getpaths|
      \pstool@process{#1}{#3}%
    }{%
      \pstool@includegraphics[#1]{#2}%
    }%
  }{%
    \pstool@process{#1}{#3}%
  }}

% \section{User commands}
%  Finally, define \cmd\pstool\ as appropriate for the mode:
\ifpdf
  \if@pstool@always@
    \let\pstool\pstool@alwaysprocess
    \WithSuffix\def\pstool!{\pstool@alwaysprocess}
    \WithSuffix\def\pstool*{\pstool@alwaysprocess}
  \else\if@pstool@never@
    \let\pstool\pstool@neverprocess
    \WithSuffix\def\pstool!{\pstool@neverprocess}
    \WithSuffix\def\pstool*{\pstool@neverprocess}
  \else
    \let\pstool\pstool@maybeprocess
    \WithSuffix\def\pstool!{\pstool@neverprocess}
    \WithSuffix\def\pstool*{\pstool@alwaysprocess}
  \fi\fi
\else
  \let\pstool\pstool@neverprocess
  \WithSuffix\def\pstool!{\pstool@neverprocess}
  \WithSuffix\def\pstool*{\pstool@neverprocess}
\fi

% \section{The figure processing}

% \cmd\ip@lastelement\ is the filename of the figure stripped of its path (if any)
\def\pstool@jobname{\ip@lastelement\pstool@suffix}

% And this is the main macro.
\newcommand\pstool@process[2]{%
  \pstool@echo@verbose{^^J=== pstool: begin processing ===}%
  \pstool@write@processfile{#1}{\ip@directpath\ip@lastelement}{#2}%
  \pstool@exe{auxiliary process: \ip@lastelement\space}
    {./}{latex
      -shell-escape
      -output-format=dvi
      -output-directory="\ip@directpath"
      -interaction=\pstool@mode\space
      \pstool@latex@opts\space
          "\pstool@jobname.tex"}%
% Execute dvips in quiet mode if latex is not run in (non/error)stop mode:
  \pstool@exe{dvips}{\ip@directpath}{%
    dvips \if@pstool@verbose@\else -q \fi -Ppdf \pstool@dvips@opts\space "\pstool@jobname.dvi"}%
  \if@pstool@pdfcrop@
    \pstool@exe{ps2pdf}{\ip@directpath}{%
      ps2pdf \pstool@pspdf@opts\space "\pstool@jobname.ps" "\pstool@jobname.pdf"}%
    \pstool@exe{pdfcrop}{\ip@directpath}{%
      pdfcrop \pstool@pdfcrop@opts\space "\pstool@jobname.pdf" "\ip@lastelement.pdf"}%
  \else
    \pstool@exe{ps2pdf}{\ip@directpath}{%
      ps2pdf \pstool@pspdf@opts\space "\pstool@jobname.ps" "\ip@lastelement.pdf"}%
  \fi
  \pstool@echo{^^J=== pstool: end processing ===^^J}%
  \pstool@endprocess{%
    \pstool@cleanup
    \pstool@includegraphics[#1]{\ip@directpath\ip@lastelement}}}

% The file that is written for processing is set up to read the preamble of the original document and set the graphic on an empty page (cropping to size is done either here with \pkg{preview} or later with \pkg{pdfcrop}).
\def\pstool@write@processfile#1#2#3{%
    \immediate\openout\pstool@out #2\pstool@suffix.tex\relax
    \immediate\write\pstool@out{%
      \noexpand\pdfoutput=0^^J% force DVI mode if not already
%
% Input the main document; redefine the document environment so only the preamble is read:
      \unexpanded{%
        \let\origdocument\document^^J%
        \let\EndPreamble\endinput^^J%
        \def\document{\endgroup\endinput}^^J}%
      \noexpand\input{\jobname}^^J%
%
% Now the preamble of the process file: (restoring document's original meaning; empty \cmd\pagestyle\ removes the page number)
      \if@pstool@pdfcrop@\else
        \noexpand\usepackage[active,tightpage]{preview}^^J%
      \fi
      \unexpanded{%
        \let\document\origdocument^^J%
        \pagestyle{empty}^^J}%
%
% And the document body to place the graphic on a page of its own:
      \unexpanded{%
        \begin{document}^^J%
        \centering\null\vfill^^J}%
      \if@pstool@pdfcrop@\else
        \noexpand\begin{preview}^^J%
      \fi
      \unexpanded{#3^^J}% this is the "psfrag" material
      \noexpand\includegraphics[#1]{\ip@lastelement}^^J%
      \if@pstool@pdfcrop@\else
        \noexpand\end{preview}^^J%
      \fi
      \unexpanded{%
        \vfill\end{document}}^^J%
      }%
    \immediate\closeout\pstool@out}

\def\pstool@cleanup{%
  \@for\@ii:=\pstool@rm@files\do{%
    \pstool@rm{\ip@directpath}{\pstool@jobname\@ii}%
}}

\providecommand\EndPreamble{}

% \section{User commands}
%
% These all support the suffixes |*| and |!|, so each user command is defined as a wrapper to \cmd\pstool.

% for EPS figures with psfrag:
\newcommand\psfragfig[2][]{\pstool@psfragfig{#1}{#2}{}}
\WithSuffix\newcommand\psfragfig*[2][]{\pstool@psfragfig{#1}{#2}{*}}
\WithSuffix\newcommand\psfragfig![2][]{\pstool@psfragfig{#1}{#2}{!}}

% Parse optional \<input definitions>
\newcommand\pstool@psfragfig[3]{%
  \@ifnextchar\bgroup{%
    \pstool@@psfragfig{#1}{#2}{#3}%
  }{%
    \pstool@@psfragfig{#1}{#2}{#3}{}%
  }%
}

% Search for both \<filename> and \<filename>|-psfrag| inputs.
%
% |#1|: possible graphicx options\\
% |#2|: graphic name (possibly with path)\\
% |#3|: |\pstool| suffix (i.e., |!| or |*| or empty\\
% |#4|: possible \pkg{psfrag} macros
\newcommand\pstool@@psfragfig[4]{%
  % Find the.eps file to use.
  \IfFileExists{#2-psfrag.eps}{%
    \edef\pstool@eps{#2-psfrag}%
    \OnlyIfFileExists{#2.eps}{%
      \PackageWarning{pstool}{Graphic "#2.eps" exists but "#2-psfrag.eps" is being used}%
    }%
  }{%
    \IfFileExists{#2.eps}{%
      \edef\pstool@eps{#2}%
    }{%
      \PackageError{pstool}{%
        No graphic "#2.eps" or "#2-psfrag.eps" found%
      }{%
        Check the path and whether the file exists.%
      }%
    }%
  }%
  % Find the .tex file to use.
  \IfFileExists{#2-psfrag.tex}{%
    \edef\pstool@tex{#2-psfrag.tex}%
    \OnlyIfFileExists{#2.tex}{%
      \PackageWarning{pstool}{%
        File "#2.tex" exists that may contain macros for "\pstool@eps.eps"^^J%
        But file "#2-psfrag.tex" is being used instead.%
      }%
    }%
  }{%
    \IfFileExists{#2.tex}{%
      \edef\pstool@tex{#2.tex}%
    }{%
      \let\pstool@tex\@empty
      \PackageWarning{pstool}{%
        No file "#2.tex" or "#2-psfrag.tex" can be found
        that may contain macros for "\pstool@eps.eps"%
      }%
    }%
  }%
  \ifx\pstool@tex\@empty
    \pstool#3[#1]{\pstool@eps}{#4}%
  \else
    \expandafter\pstool@@@psfragfig\expandafter{\pstool@tex}{#3[#1]}{#4}%
  \fi
}

% Break out the separate function in order to expand "\pstool@tex" before writing it.
\newcommand\pstool@@@psfragfig[3]{%
  \pstool#2{\pstool@eps}{%
    \csname @input\endcsname{#1}%
    #3%
  }%
}

\end{filecontents}
%%%%%%%%%1%%%%%%%%%2%%%%%%%%%3%%%%%%%%%4%%%%%%%%%5




% Conditionally compile the documentation & generate the .ins file:
\providecommand\pstoolCompile{Y}
\if\pstoolCompile N
  \expandafter\endinput
\fi


\begin{filecontents*}{pstool.ins}
%&latex
\def\pstoolCompile{N}
\input pstool.tex
\csname@@end\endcsname
\end{filecontents*}




\makeatletter
\documentclass{article}

\usepackage[rm,medium]{titlesec}

\usepackage{bigfoot,xcolor}
\usepackage[colorlinks,linktocpage]{hyperref}
\usepackage{gmdoc}
\usepackage{gmverb}
\dekclubs
\stanzaskip=\bigskipamount 
\CodeSpacesGrey

\usepackage{tocloft,varwidth}
\setcounter{tocdepth}{1}
\def\tocwidthA{0.45}
\def\tocwidthB{0.45}
\def\cftpartfont{\scshape}
\def\cftsecfont{\small}
\cftbeforesecskip=0pt
\def\cftpartleader{}
\def\cftpartafterpnum{\cftparfillskip}
\def\cftsecleader{}
\def\cftsecafterpnum{\cftparfillskip}

\let\pkg\textsf
\def\pkgopt#1{\texttt{[#1]}}

\def\PDF{\textsc{pdf}}
\def\PS{\textsc{ps}}
\def\DVI{\textsc{dvi}}
\def\EPS{\textsc{eps}}

\usepackage{pstool}
\usepackage[T1]{fontenc}
\usepackage{microtype}
\usepackage{lmodern}
\usepackage[sc,osf]{mathpazo}
\linespread{1.1}
\frenchspacing

\GetFileInfo{pstool.sty}
\begin{document}
{\addtocontents{toc}{\protect\begin{varwidth}[t]{\tocwidthA\linewidth}}}

\title{The \pkg{pstool} package}
\author{
  \normalsize Concept by Zebb Prime\\
  \normalsize Package by Will Robertson\footnote{\texttt{wspr81@gmail.com}}}
\date{\fileversion\qquad\filedate}

\maketitle

\begin{abstract}
This package defines the \cmd\psfragfig\ user command for including \EPS\ files that use \pkg{psfrag} features in a pdf\/\LaTeX\ document. The command \cmd\pstool\ can be used to define other commands with similar behaviour.
\end{abstract}

\tableofcontents

\part{User documentation}

\section{Introduction}

While directly producing \PDF\ output with pdf\/\LaTeX\ is a great improvement in many ways over the `old method' of \DVI$\to$\PS$\to$\PDF, it loses the ability to interface with a generic PostScript workflow, used to great effect in numerous packages, most notably \pkg{PSTricks} and \pkg{psfrag}.

Until now, the best way to use these packages while running pdf\/\LaTeX\ has been to use the \pkg{pst-pdf} package, which processes the entire document through a filter, sending the relevant PostScript environments (only) through a single pass of \LaTeX\ producing \DVI$\to$\PS$\to$\PDF. The resulting \PDF\ versions of each graphic are then included into the pdf\/\LaTeX\ document in a subsequent compilation. The \pkg{auto-pst-pdf} package provides a wrapper to perform all of this automatically.

The disadvantage with this method is that for every document compilation, \emph{every} graphic must be re-processed. The \pkg{pstool} package uses a different approach to allow each graphic to be processed only as needed, speeding up and simplifying the typesetting of the main document.

At present this package is designed solely as a replacement for \pkg{pst-pdf} in the r\^ole of supporting the \pkg{psfrag} package (which it loads) in pdf\/\LaTeX. 

More flexible usage to provide a complete replacement for \pkg{pst-pdf} (e.g., supporting the \cs{begin\{postscript\}} environment) is planned for a later release. If you simply need to automatically convert plain \EPS\ files to \PDF, I recommend using the \pkg{epstopdf} package with the \pkgopt{update,prepend} package options (\pkg{epstopdf} and \pkg{pstool} should be completely compatible).

\section{Getting started}

Load the package as usual; no package options are required by default, but there are a few useful options described later in section~\ref{pkgopt}. Note that you do not need to load \pkg{psfrag} separately.

The generic command provided by this package is
\begin{center}
  \cmd\pstool\ \arg[graphicx options] \arg{filename} \arg{input definitions}
\end{center}
It converts the graphic \<filename>|.eps| to \<filename>|.pdf| through a unique \DVI$\to$\PS$\to$\PDF\ process for each graphic, using the preamble of the main document. The resulting graphic is then inserted into the document, with optional \<graphicx options>. The third argument to \cmd\pstool\ allows arbitrary \<input definitions> (such as \cmd\psfrag\ directives) to be inserted before the figure as it is processed. 

The command \cmd\pstool\ can take an optional |*| or |!| suffix to change the behaviour of the command:
\begin{description}
\item[\cs{pstool}] Process the graphic \<filename>|.eps| if \<filename>|.pdf| does not already exist, or if the \EPS\ file is \emph{newer} than the \PDF;
\item[\cs{pstool*}] Always process this figure; and,
\item[\cs{pstool!}] Never process this figure.
\end{description}
The behaviour in these three cases can be overridden globally by the package option \pkgopt{process} as described in section~\ref{process}.

It is useful to define higher-level commands based on \cmd\pstool\ for including specific types of \EPS\ graphics that take advantage of \pkg{psfrag}. As an example, this package defines the following command, which also supports the |*| or |!| suffixes described above.
\begin{description}
\item[{\cmd\psfragfig\arg[opts]\arg{filename}}] 
This is the catch-all macro to support a wide range of graphics naming schemes. It inserts an \EPS\ file named either \<filename>|-psfrag.eps| or \<filename>|.eps| (in that order of preference), and uses \pkg{psfrag} definitions contained within either \<filename>|-psfrag.tex| or \<filename>|.tex|. 

This command can be used to insert figures produced by the \textsc{Mathematica} package \pkg{MathPSfrag} or the \textsc{Matlab} package \pkg{matlabfrag}. \cmd\psfragfig\ also accepts an optional braced argument as shown next.

\item[{\cmd\psfragfig\arg[opts]\arg{filename}\arg{input definitions}}] 
As above, but inserts the arbitrary code \<input definitions>, which will usually be used to define new or override existing \pkg{psfrag} commands.
\end{description}

\section{Package options}
\label{pkgopt}

\subsection{Forcing/disabling graphics processing}
\label{process}

While the suffixes |*| and |!| can be used to force or disable (respectively) the processing of each individual graphic, sometimes we want to do this on a global level. The following package options  override \emph{all} \cmd\pstool\ (and related) macros: 
\begin{description}
\item[\pkgopt{process=auto}] This is the default mode as described in the previous section, in which graphics with suffixes are only (re-)processed if the \EPS\ file is newer or the \PDF\ file does not exist;
\item[\pkgopt{process=all}] Suffixes are ignored and all \cmd\pstool\ graphics are processed;
\item[\pkgopt{process=none}] Suffixes are ignored and no \cmd\pstool\ graphics are processed.\footnote{If \pkg{pstool} is loaded in a \LaTeX\ document in \DVI\ mode, this is the option that is used since no external processing is required for these graphics.}
\end{description}

Also note that it would be nice to detect the age of files other than the \EPS\ and \PDF\ graphics in order to affect the processing decisions. This is planned for a possible future release.

\subsection{Cropping graphics}
The default option \pkgopt{crop=preview} selects the \pkg{preview} package to crop graphics to the appropriate size for each auxiliary process.

However, when an inserted label protrudes from the natural bounding box of the figure, or when the original bounding box of the figure is wrong, the \pkg{preview} package will not always produce a good result (with parts of the graphic trimmed off the edge). A robust method to solve this problem is to use the \pkg{pdfcrop} program instead.%
\footnote{\pkg{pdfcrop} requires a Perl installation under Windows, freely available from \url{http://www.activestate.com/Products/activeperl/index.plex}} This can be activated in \pkg{pstool} with the \pkgopt{crop=pdfcrop} package option. 

In the future I plan to also support  \pkg{epstool} for doing the same thing.

\subsection{Temporary files \& cleanup}
Each figure that is processed spawns an auxiliary \LaTeX\ compilation through \DVI$\to$\PS$\to$\PDF. This process is named after the name of the figure with an appended string suffix; the default is \pkgopt{suffix=\{-pstool\}}. All of these suffixed files are ``temporary'' in that they may be deleted once they are no longer needed.

As an example, if the figure is called |ex.eps|, the files that are created are |ex-pstool.tex|, |ex-pstool.dvi|, \dots. The \pkgopt{cleanup} package option declares via a list of filename suffixes which temporary files are to be deleted after processing.

The default is \pkgopt{cleanup=\{.tex,\,.dvi,\,.ps,\,.pdf,\,.log,\,.aux\}}. To delete none of the temporary files, choose \pkgopt{cleanup=\{\}} (useful for debugging).

\subsection{Interaction mode of the auxiliary processes}
Each graphic echoes the output of its auxiliary process to the console window; unless you are trying to debug errors there is little interest in seeing this information. The behaviour of these auxiliary processes are governed globally by the \pkgopt{mode} package option, which takes the following parameters:
\begin{description}
\item[\pkgopt{mode=batch}] hide almost all of the \LaTeX\ output (\emph{default});
\item[\pkgopt{mode=nonstop}] echo all \LaTeX\ output but continues right past any errors; and 
\item[\pkgopt{mode=errorstop}] prompt for user input when errors in the source are encountered.
\end{description}
These three package options correspond to the \LaTeX\ command line options \texttt{-interaction=batchmode}, \texttt{=nonstopmode}, and \texttt{=errorstopmode}, respectively. When \pkgopt{mode=batch} is activated, then \pkg{dvips} is also run in `quiet mode'.

\subsection{Auxiliary processing command line options}
The command line options passed to each program of the auxiliary processing can be changed with the following package options:
\begin{description}
\item[\pkgopt{latex-options}] ;
\item[\pkgopt{dvips-options}] ;
\item[\pkgopt{ps2pdf-options}] ; and,
\item[\pkgopt{pdfcrop-options}] .
\end{description}
For the most part these will be unnecessary, although passing the correct options to |ps2pdf| can sometimes be a little obscure. For example, I use the following for generating figures in my thesis:\par
{\centering|ps2pdf-options={-dCompatibilityLevel=1.4 -dPDFSETTINGS=/prepress}|\par}
I believe this incantation forces fonts to be embedded within the individual figure files, without which some printers and \textsc{pdf} viewers have trouble with the textual labels.

\section{Miscellaneous details}

\subsection{The \cmd\EndPreamble\ command}

At present, \pkg{pstool} scans the preamble of the main document by redefining |\begin{document}|, but this is rather fragile  because many classes and packages do their own redefining which overwrites \pkg{pstool}'s attempt. In this case, place the command\par
{\centering|\EndPreamble|\par}\noindent
where-ever you'd like the preamble in the auxiliary processing to end (although is must be placed before |\begin{document}| for obvious reasons). This is also handy to bypass anything in the preamble that will never be required for the figures but which will slow down or otherwise conflict with the auxiliary processing.

\subsection{Cross-reference limitations}

The initial release of this package does not support cross-references within the \pkg{psfrag} labels of the included graphics. (If, say, you wish to refer to an equation number within a figure.) A future release of \pkg{pstool} will hopefully lift this limitation.

\subsection{A note on file paths}

\pkg{pstool} does its best to ensure that you can put image files where-ever you like and the auxiliary processing will still function correctly. In order to ensure this,  the external |pdflatex| compilation uses the |-output-directory| feature of \pdfTeX. This command line option is definitely supported on all platforms in TeX~Live~2008 and MiKTeX~2.7, but earlier distributions may not be supported.

One problem that \pkg{pstool} does not (currently) solve on its own is the inclusion of images that do not exist in subdirectories of the main document. For example, |\pstool{../Figures/myfig}| will not process by default because \pdfTeX\ usually does not have permission to write into folders that are higher in the heirarchy than the main document. This can be worked around presently in two different ways: (although maybe only for Mac~OS~X and Linux)
\begin{enumerate}
\item Give |pdflatex| permission to write anywhere with the command:\\
|openout_any=a pdflatex ...|
\item Create a symbolic link in the working directory to a point higher in the path: |ln -s ../../PhD ./PhD|, for example, and then refer to the graphics through this symbolic link.
\end{enumerate}
I hope to directly solve this problem in the future by using a caching folder for the auxiliary processing in such cases.

\section{Package information}
The most recent publicly released version of \pkg{pstool} is available at \textsc{CTAN}:\\
\centerline{\url{http://tug.ctan.org/pkg/pstool/}}\\ 
Historical and developmental versions are available at GitHub:\\ \centerline{\url{http://github.com/wspr/pstool/}}\\
While general feedback at \url{wspr81@gmail.com} is welcomed, specific bugs should be reported through the bug tracker at FogBugz: \url{https://wspr.fogbugz.com/} (click `\textsc{tasks}: Enter a New Case').

\subsection{Licence}

This package is freely modifiable and distributable under the terms and conditions of the \LaTeX\ Project Public Licence, version 1.3c or greater (your choice). The latest version of
this license is available at: \url{http://www.latex-project.org/lppl.txt}. This work is maintained by \textsc{Will Robertson}.

{\addtocontents{toc}{\protect\end{varwidth}\protect\hfill}}
{\addtocontents{toc}{\protect\begin{varwidth}[t]{\protect\tocwidthB\protect\linewidth}}}
\clearpage
\part{Implementation}
\parindent=0pt
\DocInput{pstool.sty}

{\addtocontents{toc}{\protect\end{varwidth}}}
\end{document}